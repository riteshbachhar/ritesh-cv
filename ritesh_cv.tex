\documentclass[10pt, letterpaper]{article}

% --------- Packages --------
\usepackage[
    ignoreheadfoot,                       % set margins without considering header and footer
    top=2 cm,                             % seperation between body and page edge from the top
    bottom=2 cm,                          % seperation between body and page edge from the bottom
    left=2 cm,                            % seperation between body and page edge from the left
    right=2 cm,                           % seperation between body and page edge from the right
    footskip=1.0 cm,                      % seperation between body and footer
    % showframe                           % Uncomment for debugging page geometry 
]{geometry}                               % for adjusting page geometry

\usepackage[explicit]{titlesec}                  % Customizing section titles
\usepackage{tabularx}                            % Create tables with fixed width columns
\usepackage{array}                               % Required by tabularx
\usepackage[dvipsnames]{xcolor}                  % For defining and using color
\usepackage{enumitem}                            % Customize lists
\usepackage{fontawesome5}                        % Use icons
\usepackage{academicons}                         % Academic icons
\usepackage{amsmath}                             % Math support
\usepackage[
    pdftitle={Ritesh Bachhar's CV},
    pdfauthor={Ritesh Bachhar},
    pdfcreator={LaTeX with RenderCV},
    colorlinks=true,
    urlcolor=primaryColor                       % Hyperlink color set to primary color
]{hyperref}                                     % Links, metadata and bookmarks
\usepackage[pscoord]{eso-pic}                   % Floating text on the page
\usepackage{calc}                               % For calculating lengths
\usepackage{bookmark}                           % Bookmarks in the PDF
\usepackage{lastpage}                           % Get the total number of pages
\usepackage{changepage}                         % One column entries (adjustwidth environment)
\usepackage{paracol}                            % Two and Three column entries
\usepackage{ifthen}                             % Conditional statements
\usepackage{needspace}                          % Avoid page break after section titles
\usepackage{iftex}                              % Check for pdflatex, xetex or luatex

% -------- Primary Color Definition --------
\definecolor{primaryColor}{RGB}{0, 79, 144} % Define the primary color for consistent styling

% -------- Machine-Readable PDF Settings --------
\ifPDFTeX
    \input{glyphtounicode}
    \pdfgentounicode=1
    \usepackage[T1]{fontenc}
    \usepackage[utf8]{inputenc}
    \usepackage{lmodern}
\fi

\usepackage[default, type1]{sourcesanspro}    % Use Source Sans Pro font

% -------- General Settings --------
\AtBeginEnvironment{adjustwidth}{\partopsep0pt}         % Remove space before adjustwidth
\pagestyle{empty}                                       % No header or footer
\setcounter{secnumdepth}{0}                             % No section numbering
\setlength{\parindent}{0pt}                             % No paragraph indentation
\setlength{\topskip}{0pt}                               % No top skip
\setlength{\columnsep}{0.15cm}                          % Column seperation

% -------- Footer Customization --------
\makeatletter
\let\ps@customFooterStyle\ps@plain                      % Copy the plain style to customFooterStyle
\patchcmd{\ps@customFooterStyle}{\thepage}{
    \color{gray}\textit{\small Ritesh Bachhar - Page \thepage{} of \pageref*{LastPage}}
}{}{}
\makeatother
\pagestyle{customFooterStyle}                           % Apply the custom footer style

% -------- Section Title Formatting --------
\titleformat{\section}{
    \needspace{4\baselineskip}                          % Avoid page break after title
    \Large\color{primaryColor}                          % Large, primary-colored text
}{
}{
}{
    % print bold title, give 0.15 cm space and draw a line of 0.8 pt thickness
    % from the end of the title to the end of the body
    \textbf{#1}\hspace{0.15cm}\titlerule[0.8pt]\hspace{-0.1cm}
}[] 

% -------- Section Title Spacing --------
\titlespacing{\section}{-1pt}{0.3 cm}{0.2 cm}           % left space, top space, bottom space


% -------- Custom Bullet Point Formatting --------
% New environment for highlights
\newenvironment{highlights}{
    \begin{itemize}[
        topsep=0.10 cm,
        parsep=0.10 cm,
        partopsep=0pt,
        itemsep=0pt,
        leftmargin=0.4 cm + 10pt
    ]
}{
    \end{itemize}
}

% New environment for highlights for bullet entries
\newenvironment{highlightsforbulletentries}{
    \begin{itemize}[
        topsep=0.10 cm,
        parsep=0.10 cm,
        partopsep=0pt,
        itemsep=0pt,
        leftmargin=10pt
    ]
}{
    \end{itemize}
}

% -------- Custom Environments --------
% New environment for one column entries
\newenvironment{onecolentry}{
    \begin{adjustwidth}{
        0.2 cm + 0.00001 cm
    }{
        0.2 cm + 0.00001 cm
    }
}{
    \end{adjustwidth}
}

% New environment for two column entries
\newenvironment{twocolentry}[2][]{
    \onecolentry
    \def\secondColumn{#2}
    \setcolumnwidth{\fill, 4.5 cm}
    \begin{paracol}{2}
}{
    \switchcolumn \raggedleft \secondColumn
    \end{paracol}
    \endonecolentry
}

% New environment for three column entries
\newenvironment{threecolentry}[3][]{
    \onecolentry
    \def\thirdColumn{#3}
    \setcolumnwidth{2.5 cm, \fill, 4.5 cm}
    \begin{paracol}{3}
    {\raggedright #2} \switchcolumn
}{
    \switchcolumn \raggedleft \thirdColumn
    \end{paracol}
    \endonecolentry
}

% New environment for the header
\newenvironment{header}{
    \setlength{\topsep}{0pt}\par\kern\topsep\centering\color{primaryColor}\linespread{1.5}
}{
    \par\kern\topsep
}

% -------- Custom Commands --------
\newcommand{\publication}[4]{%
    \textbf{#1} \\ % Title
    \textit{#2} \\ % Authors
    \textit{#3} \\ % Journal and Year
    DOI: \href{#4}{#4} % DOI
}


% --- Last Updated Text ---
\newcommand{\placelastupdatedtext}{% \placetextbox{<horizontal pos>}{<vertical pos>}{<stuff>}
  \AddToShipoutPictureFG*{% Add <stuff> to current page foreground
    \put(
        \LenToUnit{\paperwidth-2 cm-0.2 cm+0.05cm},
        \LenToUnit{\paperheight-1.0 cm}
    ){\vtop{{\null}\makebox[0pt][c]{
        \small\color{gray}\textit{Last updated in December 2024}\hspace{\widthof{Last updated in December 2024}}
    }}}%
  }%
}%


% --- Hyperlink Redefinition ---
\let\hrefWithoutArrow\href
\renewcommand{\href}[2]{\hrefWithoutArrow{#1}{\ifthenelse{\equal{#2}{}}{ }{#2 }\raisebox{.15ex}{\footnotesize \faExternalLink*}}}


% ----- Main Document -----
\begin{document}

\newcommand{\AND}{\unskip                       % Define a custom horizontal separator command
    \cleaders\copy\ANDbox\hskip\wd\ANDbox
    \ignorespaces
}

\newsavebox\ANDbox                              % Create a reusable empty box for the separator
\sbox\ANDbox{}                                  % Define the box (currently empty; customizable later)

\placelastupdatedtext                           % Add "Last updated" text to the document

% ---- Header ----
\begin{header}

  % ---- Name ----
  \fontsize{30 pt}{30 pt}
  \textbf{Ritesh Bachhar}
  \vspace{0.3 cm}
  \normalsize % Set normal font size for the following text

  % ---- Location ----
  \mbox{{\footnotesize\faMapMarker*}\hspace*{0.13cm}RI, USA}
  \kern 0.25 cm%
  \AND%
  \kern 0.25 cm%
  % ---- Contact Information ----
  \mbox{\hrefWithoutArrow{mailto:riteshbachhar@uri.edu}{{\footnotesize\faEnvelope[regular]}\hspace*{0.13cm}riteshbachhar@uri.edu}}%
  \kern 0.25 cm%
  \AND%
  \kern 0.25 cm%
  \mbox{\hrefWithoutArrow{mailto:riteshbachhar@gmail.com}{{\footnotesize\faEnvelope[regular]}\hspace*{0.13cm}riteshbachhar@gmail.com}}%
  % \mbox{\hrefWithoutArrow{tel:+90-541-999-99-99}{{\footnotesize\faPhone*}\hspace*{0.13cm}0541 999 99 99}}%
  \kern 0.25 cm%
  % ---- Personal website ----
  \AND%
  \kern 0.25 cm%
  \mbox{\hrefWithoutArrow{https://riteshbachhar.com}{{\footnotesize\faLink}\hspace*{0.13cm}riteshbachhar.com}}%
  \kern 0.25 cm%
  \AND%
  % ---- LinkedIn and GitHub ----
  \kern 0.25 cm%
  \mbox{\hrefWithoutArrow{https://inspirehep.net/authors/2160749?ui-citation-summary=true}{{\footnotesize\aiInspire}\hspace*{0.13cm}InspireHEP}}%
  \kern 0.25 cm%
  \AND%
  \kern 0.25 cm%
  \mbox{\hrefWithoutArrow{https://scholar.google.com/citations?user=rqhoVfgAAAAJ&hl=en&authuser=1}{{\footnotesize\aiGoogleScholar}\hspace*{0.13cm}Google Scholar}}%
  \kern 0.25 cm%
  \AND%
  \kern 0.25 cm%
  \mbox{\hrefWithoutArrow{https://github.com/riteshbachhar}{{\footnotesize\faGithub}\hspace*{0.13cm}GitHub}}%
\end{header}

\vspace{0.3 cm - 0.3 cm}

% ----- Research Interests -----
\section{Research Interests}
  \begin{onecolentry}
    General relativity~(GR), Surrogate models of gravitational waves, Gravitational wave parameter estimation, Black hole merger phenomenology, point-particle black hole perturbation theory
  \end{onecolentry}

% ---------- Education ----------
\section{Education}
  \begin{threecolentry}{\textbf{Ph.D.}}{
    September 2021 - Present
  }
    University of Rhode Island, Kingston, RI, USA
    \begin{highlights}
      % \item CPI: 3.99/4.0
      \item Advisor: Dr. Gaurav Khanna
    \end{highlights}
  \end{threecolentry}

\begin{threecolentry}{\textbf{M.Sc. Physics}}{
    July 2018 - June, 2020
  }
    Indian Institute of Technology Bombay, Mumbai, India
  \begin{highlights}
    % \item CPI: 9.17/10
    \item Advisor: Dr. Varun Bhalerao
    \item Thesis: Phase resolved analysis of Centaurus X-3
  \end{highlights}
\end{threecolentry} 

\begin{threecolentry}{\textbf{B.Sc. Physics}}{
  July 2014 - April 2017
}
  Scottish Church College, Kolkata, India
  % \begin{highlights}
  %     \item \textbf{Coursework:} Classical Mechanics, Quantum Mechanics, Electrodynamics, Statistical Mechanics, Thermodynamics, Mathematical Physics, Electronics, and Optics
  % \end{highlights}
\end{threecolentry}


% ---------- Awards & Honors ----------
\section{Awards} 
\begin{onecolentry}
  \begin{highlightsforbulletentries}
    \item \textbf{Dean's Fellowship}, University of Rhode Island

    \item NASA travel grant of \$1500 for attending the 15th LISA Symposium, Dublin, Ireland in 2024.

    \item \textbf{Bhavesh Gandhi Memorial Prize}(2019-20), IIT Bombay, for best M.Sc. thesis.

    \item Qualified \textbf{CSIR NET} June 2019 with All India Rank~\textbf{(AIR) 66} and eligible for Junior Research Fellowship.
  \end{highlightsforbulletentries}
\end{onecolentry}


% ---------- Employment ----------
\section{Employment} 
\begin{onecolentry}
  \begin{highlightsforbulletentries}
    \item \textbf{Dean's Fellow}, University of Rhode Island, Fall 2024 - Summer 2025
    \item \textbf{Research Assistant}, University of Rhode Island, Fall 2022 - Spring 2024
    \item \textbf{Teaching Assistant}, University of Rhode Island, Fall 2021 - Spring 2022
  \end{highlightsforbulletentries}
\end{onecolentry}


% ---------- Publications ----------
\section{Publications}

\begin{enumerate}
    \item \publication{Incorporating waveform calibration error in gravitational-wave modeling and inference for SEOBNRv4}
            {\textbf{Ritesh Bachhar}, Michael Pürrer, Stephen R. Green}
            {arXiv:2410.17168, October 2024}
            {https://arxiv.org/abs/2410.17168}

    \item \publication{Gravitational wave surrogate model for spinning, intermediate mass ratio binaries based on perturbation theory and numerical relativity}
          {Katie Rink, \textbf{Ritesh Bachhar}, Tousif Islam \textit{et al.}}
          {arXiv:2407.18319, July 2024}
          {https://arxiv.org/abs/2407.18319}

    \begin{samepage}        
    \item \publication{Binary Black Hole Coalescence Phenomenology from Numerical Relativity}
            {Richard Price, \textbf{Ritesh Bachhar}, Gaurav Khanna}
            {arXiv:2312.15885; December 2023}
            {https://arxiv.org/abs/2312.15885}
    \end{samepage}
    
    \item \publication{Angular Momentum for Black Hole Binaries in Numerical Relativity}
            {\textbf{Ritesh Bachhar}, Richard Price, Gaurav Khanna}
            {PRD 108,064019; September 2023}
            {https://journals.aps.org/prd/abstract/10.1103/PhysRevD.108.064019}

    \item \publication{Timing and spectral studies of Cen X-3 in multiple luminosity states using AstroSat}
          {\textbf{Ritesh Bachhar}, Gayathri Raman, Varun Bhalerao \textit{et al.}}
          {MNRAS, 517, 4138; October 2022}
          {https://academic.oup.com/mnras/article/517/3/4138/6760011}
\end{enumerate}


% ---------- Research Experience ----------
\section{Research Experience}
\begin{twocolentry}{September 2021 - Present}{\textbf{Surrogate modeling of gravitational waves}}
    \begin{highlights}
        \item Studied point-particle black hole perturbation theory
        \item Developing reduced-order surrogate models for gravitational waves from binary black holes
        \item Combining numerical relativity and perturbation theory to model the gravitational waveforms of spinning black holes
        \item Modeling the gravitational waveforms of intermediate mass ratio inspirals
    \end{highlights}
\end{twocolentry}

\vspace{0.2 cm}

\begin{twocolentry}{July 2023 - Present}{\textbf{Systematics in gravitation wave modeling}}
    \begin{highlights}
        \item Studied the impact of waveform calibration errors on gravitational wave parameter estimation in the context of \texttt{SEOBNRv4}
        \item Incorporated the waveform calibration error in the gravitation wave modeling
        \item Performed parameter estimation studies on simulated data to quantify the impact of waveform calibration errors
    \end{highlights}
\end{twocolentry}

\vspace{0.2 cm}

\begin{twocolentry}{August 2022 - January 2024}{\textbf{Black hole merger phenomenology}}
    \begin{highlights}
        \item Studied the phenomenology of binary black hole mergers
        \item Integrating approximate methods to complement numerical studies
        \item Investigated the role of orbital angular momentum in the dynamics of binary black hole mergers
    \end{highlights}
\end{twocolentry}

\vspace{0.2 cm}

\begin{twocolentry}{July 2019 - January 2022}{\textbf{X-ray analysic of accerating pulsars}}
    \begin{highlights}
        \item Studied the process of X-ray emission of HMXBs, particularly Cen X-3
        \item Analyzed the phase resolved X-ray spectra of Cen X-3 using AstroSat data
        \item Observed Quasi-periodic oscillations in the X-ray light curves of Cen X-3
    \end{highlights}
\end{twocolentry}

% ---------- Teaching Experience ----------
\section{Teaching Experience}
\begin{onecolentry}
    \begin{highlightsforbulletentries}
      \item TA for PHY204, Elementary Physics II (Fall 2023), with Dr. Rob Coyne
      \begin{itemize}
        \item Assisted students with homework assignments and exam preparation by providing guidance and support to strengthen their understanding of the subject matter
      \end{itemize}

      \item TA for AST 108 and AST118H (Fall 2021 - Spring 2022) with Prof. Douglas Gobeille
      \begin{itemize}
        \item Led various doubt-clearing sessions to enhance students' understanding of the course material
        \item Organized observation sessions to guide students in identifying celestial objects, including planets, stars, and constellations
        \item Conducted telescopic observations of solar phenomena, such as sunspots and solar flares
      \end{itemize}   
    \end{highlightsforbulletentries}
\end{onecolentry}


% ---------- Talks and Conferences ----------
\section{Talks and Conferences}    
\begin{highlightsforbulletentries}
    \item \begin{twocolentry}
        {November 2024}{University of Glasgow, Student Seminar, Online}
        \\\textit{Incorporating Waveform Uncertainties in Effective-One-Body Models for Accurate Gravitational Wave Parameter Estimation}
    \end{twocolentry}

    \item 
    \begin{twocolentry}{April 2024}{American Physical Society (APS) April Meeting}
        \\\textit{Gravitational wave inference with marginalization over waveform uncertainty}
    \end{twocolentry}

    \item \begin{twocolentry}
        {January 2024}{InterDisciplinary made EAsy (IDEA), Brown University}
        \\\textit{Gravitational waveform models for intermediate mass ratio binary black hole systems: Extending the reach of black hole perturbation theory with numerical relativity}
    \end{twocolentry}

    \item 
    \begin{twocolentry}{June 2023}{23rd Eastern Gravity Meeting}
        \\\textit{Building surrogate model of spinning binary black hole coalescence using perturbation theory waveforms}
    \end{twocolentry}

    \item 
    \begin{twocolentry}{April 2023}{American Physical Society (APS) April Meeting}
      \\\textit{Surrogate model for gravitational waveforms from spinning binary black hole coalescence using perturbation theory}
    \end{twocolentry}
\end{highlightsforbulletentries}


% ---------- Programming Skills ----------
\section{Programming Skills}

    \begin{onecolentry}
        \textbf{Programming:} Python(NumPy, SciPy, SymPy, AstroPy and Pandas), C, Fortran
    \end{onecolentry}

    \vspace{0.2 cm}

    \begin{onecolentry}
        \textbf{Software:} Matlab, Mathematica, HEASOFT, XSPEC, and LaTe
    \end{onecolentry}


    

\end{document}